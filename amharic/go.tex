\documentclass[12pt,a4paper]{report}
\usepackage[utf8]{inputenc}
\usepackage{listings}
\usepackage{listings-golang} % import this package after listings
\usepackage{color}
\lstset{ % add your own preferences
    frame=single,
    basicstyle=\footnotesize,
    keywordstyle=\color{red},
    numbers=left,
    numbersep=5pt,
    showstringspaces=false, 
    stringstyle=\color{blue},
    tabsize=4,
    language=Golang % this is it !
}
\usepackage{fontspec}
\begin{document}
\fontspec{AbyssinicaSIL}
\noindent
\huge 
ስለዚህ መጽሐፍ
\normalsize
\vskip 0.5in
\huge 
\noindent
ፍቃድ
\vskip 0.3in

\normalsize
\noindent
The Little Go Book  በ Attribution-NonCommercial-ShareAlike 4.0 አለም አቀፍ ፍቃድ የተዘጋጀ ነው። ለዚህ መጽሐፍ ምንም አይነት ክፍያ መፈጸም የለበትም ። ይሄንን መጽሐፍ በነጻ መባዛት ፣ ማሰራጨት ፣ ይዘቱን መቀየር እንዲሁም ማስነበብ ይቻላል ። ነገር ግን ለጸሃፊው ካርል ሰጉዊን እውቅና መስጠት ይኖርቦታል ፤ መጽሐፉን ለንግድ አላማ አይጠቀሙበት።
\vskip 0.3in
\noindent
The Little Go Book  በ Attribution-NonCommercial-ShareAlike 4.0 አለም አቀፋዊ ፍቃድ የተዘጋጀ ነው። ለዚህ መጽሐፍ ምንም አይነት ክፍያ መፈጸም የለበትም ። ይሄንን መጽሐፍ በነጻ መባዛት ፣ ማሰራጨት ፣ ይዘቱን መቀየር እንዲሁም ማስነበብ ይቻላል ። ነገር ግን ለጸሃፊው ካርል ሰጉዊን እውቅና መስጠት ይኖርቦታል ፤ መጽሐፉን ለንግድ አላማ አይጠቀሙበት።
የመጽሐፉን ፍቃድ ሙሉ ዝርዝር የሚከተለው ማስፈንጠሪያ ላይ ማየት ይቻላል፤
https://creativecommons.org/licenses/by-nc-sa/4.0/
\vskip 0.3in
\noindent
\huge የመጨረሻ ስሪት
\normalsize
\vskip 0.3in
\noindent
የመጽሐፉን የቅርብ ጊዜ ምንጭ ከዚህ ማግኘት ይቻላል ፤ 

https://github.com/karlseguin/the-little-go-book
\vskip 0.3in
\huge
\noindent
መግቢያ
\normalsize
\vskip 0.3in
\noindent
ሁሌም አዲስ የፕሮግራሚንግ ቋንቋ ለመማር ስነሳ ድብልቅልቅ ስሜት ነው የሚሰማኝ ። በአንድ በኩል ፤ ቋንቋ (ፕሮግራሚንግ) ለምንሰራው ነገር በጣም መሰረታዊ ነው ፤በጣም አነስተኛዋ ለውጥ በጣም ከፍተኛ የሚባል ተጽእኖ ልትፈጥር ትችላለች። ፕሮግራም ስንጽፍ የሚገጥሙን አስተማሪ አጋጣሚዎች ወደፊት ለሚኖረን የፕሮግራሚንግ ስልት ዘላቂ የሆነ ተጽእኖ ሊፈጥሩብን እንዲሁም ከሌሎች ቋንቋዎችም የምንጠብቃቸውን ነግሮች እንድናገናዝብ ያደርጉናል። በሌላ በኩል ደግሞ የቋንቋዎች ዲዛይን በአብዛኛው ጭማሪ ጽንሰ-ሃሳቦችን ይዘው ነው የሚመጡት ። ስለዚህ አዲስ ቋንቋ ለመማር ፤ አዳዲስ ቁልፍ ቃላትን፣ የአይነት ስርዓት ፣ የኮድ ስልት ፣ ላይብረሪዎች ፣ ኮሚኒቲዎች እንዲሁም አዲስ አስተሳሰብ መለማመድ ከባድ ስራ ሊመስል ይችላል። ከሌሎች ትምህርቶች አንጻርም አዲስ ቋንቋ መማር ላይ የምናጠፋው ጊዘ የባከነ መስሎ ሊሰማን ይችላል።
\vskip 0.3in
\noindent
ይህን ካልኩ በኋላ ወደፊት ስንቀጥል ፤ አዲስ ቋንቋ በመማር ደረጃ በደረጃ ጭማሪ ለውጦችን ለመውሰድ ፍቃደኛ መሆን አለብን ፤ ምክንያቱም የፕሮግራሚንም ቋንቋዎች ለምንሰራቸው ነገሮች መሰረት ናቸው ። ለውጦቹ ቅጥልጣይ ጭማሪዎች ቢሆኑም ፤ ሰፊ አድማስ አላቸው ። በዚህም ምክንያት ምርታማነት (productivity) ፣ ተነባቢነት (readability) ፣ ውጠታማነት (performance) ፣ ተሞካሪነት (testability) ፣ የኮድ ጥገኝነት (dependency management) ፣ ጥርሰት አያያዝ (error handling) ፣ አሰናነድ (documentation) ፣ ትንታነ (profiling) ፣ ማህበረሰብ (communities) ፣ መነሻ ላይብረሪ (standard libraries) እና የመሳሰሉት ላይ ተጽእኖ ያሳርፋሉ። ምናልባትም እንደዚህ አይነት ለውጦችን እንደመራራ መድሐኒት ልንወስድ እንችላለን?
\vskip 0.3in
\noindent
ይህ ወደዋናው ጥያቀ ያመራናል ፤ ጎ ለምን አስፈለገ ? በእነ እይታ ሁለት አንገብጋቢ ምክንያቶች አሉ ። የመጀመሪያው ምክንያት ፤ በአንፃሩ ቀላል ቋንቋ ስለሆነና መነሻ ላይብረሪውም ቀላል ስለሆነ ነው ። በብዙ መልኩ ሲታይ ፤ ጎ ላይ ያሉት ለውጦች ባለፉት ሁለት አስርት አመታት በፕሮግራሚንግ ቋንቋዎች ላይ እየጨመረ የነበረውን ውስብስብነት የሚያቃልሉ ናቸው ። ለላው ምክንያት ደግሞ ለብዙ ገንቢዎች (developers) ጎ ተጨማሪ አቅም ስለሚሰጥ ነው ።
\vskip 0.3in
\noindent
ጎ የተገነባው እንደ ሲስተም ቋንቋ (ምሳለ ፦ ኦፐረቲንግ ሲስተም ፣ ዲቫይስ ድራይቨር) የ C እና የ C++ ገንቢዎችን ታሳቢ አድርጎ ነው ። ነገር ግን ጎ-ቲም (Go team) ባለው መረጃ መሰረት የ ጎ ዋና ተጠቃሚዎች የሆኑት የሲስተም ገንቢዎች ሳይሆኑ የ ትግበራ(application) ገንቢዎች ናቸው ። ለምን ? የሲስተም ገንቢዎችን ወክየ መናገር አልችልም ፤ ነገር ግን እንደነ ድረ-ገጽ ለሚገነቡ ፣ ለግልጋሎት (services) ፣ ለደስክቶፕ ትግበራዎች እና ለመሳሰሉት እየተፈጠረ ያለውን በከፍተኛ-እርከንና (higher-level) በዝቅተኛ እርከን (low-level) የሲስተም ትግበራዎች መካከል ያሉ ሲስተሞች ተፈላጊነት እያሟላ ስለሆነ ይመስለኛል ። 
\vskip 0.3in
\noindent
ምናልባትም ለመላላኪያ (messaging) ፣ ማቆያ (caching) ፣ ከፍተኛ ስለት ለሚያስፈልገው ውህብ ትንታነ ፣ ለትእዛዝ መስመር መግቢያ (command line interface) ፣ መመዝገቢያ (logging) ወይም መቆጣጠሪያ (monitoring) የተመቸ ቋንቋ ነው ልንል እንችላለን ። እነ ባለኝ የስራ ልምድ የሲስተሞች ውስብስብነት እየጨመረ ሲሀድና ኮንከረንሲ በሺዎች መቆጠር ሲጀምር ፤ ይሀንን ሊፈታ የሚችል የኢንፍራስትራክቸር ሲስተም ያስፈልጋል ። እንደ ሩቢ ወይም ፓይተን ያሉ ቋንቋዎችን በመጠቀም እንደዚህ አይነት ሲስተሞች መገንባት የተለመደ ነው ፤ ግን ለእነዚህ ሲስተሞች የጎ ቋሚ የአይነት ስርዓት (type system) እና የተሻለ ውጠታማነት የበለጠ ያሻሽላቸዋል ። እርግጥ ነው ጎ ን በመጠቀም ድረ-ገጽ መገንባት የተለመደ ነው ፤ እንደነ አስተያየት ግን ለዚህ አላማ ኖድ ወይም ሩቢ የተሻለ ተመራጭ ናቸው ብየ አምናለሁ ።
\vskip 0.3in
\noindent
ጎ የላቀ ጥቅም የሚያስገኝባቸው ለሎች ቦታዎች አሉ ። ለምሳለ ፤ ኮምፓይልድ የሆነ የ ጎ ፕሮግራም ስናስፈጽም ምንም አይነት ቅደመ ሁነታ ወይም ጥገኝነት አይኖርም ። ለማነጻጸር ያህል ለምሳለ የሩቢ ወይም የጃቫ ፕሮግራሞች የተጫነው የሩቢ ወይም የ JVM ስሪት ላይ ይወሰናሉ ። በዚህ ምክንያት የጎ ፕሮግራሚንግ ቋንቋ ለትእዛዝ-መስመር መግቢያ(command-line interface) ፕሮግራሞች እና ለለሎች ኮምፓይልድ ሆነው ለሚሰራጩ መገልገያዎች በጣም ተመራጭ እየሆነ ነው ።
\vskip 0.3in
\noindent
በግልፅ ለማስቀመጥ የጎ ቋንቋን መማር ወቅታዊና ጠቃሚ ነው ። ለመማርም ሆነ ቋንቋውን ለመካን ብዙ ጊዘ አይፈጅም ፤ በአጭር ጊዘ ውስጥ ተግባራዊ የሚሆን ክህሎት ታዳብራላቹህ ።
\vskip 0.3in
\noindent
\huge ማስታወሻ ከፀሐፊው
\normalsize
\vskip 0.3in
\noindent
ይሀንን መፅሐፍ ለመጻፍ ስነሳ በሁለት ምክንያቶች የተነሳ ሳቅማማ ነበር ። የመጀመሪያው ፤ የ ጎ ሰነድ በተለይም Effective Go በጣም ምርጥ የሆነ መማሪያ ስለሆነ ነው ።
\vskip 0.3in
\noindent
ለላው ደግሞ ስለፕሮግራሚንግ ቋንቋ መፅሐፍ ማዘጋጀት ብዙም ምቾት ስላልሰጠኝ ነበር ። ከዚህ በፊት The Little MongoDB Book የሚለውን መፅሐፍ ሳዘጋጅ ፤ አንዛኛዎቹ አንባቢዎች ቢያንስ ስለ ሪለሽናል ዳታበዝ መሰረታዊ እውቀት ይኖራቸዋል ብየ ስላሰብኩ ቀላል ነበር ።  The Little Redis Book መፅሐፍ ሳዘጋጅ ደግሞ ከ ቁልፍና ዋጋ መዝገብ (key value store) መሰረታዊ እውቀት መነሳት በቂ ነበር ።
\vskip 0.3in
\noindent
ከዚህ ቀጥሎ የሚመጡትን የዚህን መፅሐፍ ምዕራፎች ሳስብ ግን ፤ ምን ያህል መሰረታዊ መነሻ እውቀት ያስፈልጋል የሚለውን ለመለካት አልቻልኩም ። ለምሳለ interface የሚባለውን ፅንሰ-ሃሳብ ለማየት ምን ያህል በጥልቅት መሀድ ያስፈልጋል ? ለአንዳንድ ሰው አዲስ የሆነ ፅንሰ-ሃሳብ ሲሆን ፤ ለለላው አንባቢ ደግሞ interface ጎ ውስጥ አለ ብሎ ማለፍ በቂ ለሆን ይችላል ። ስለዚህ ይህን መፅሐፍ ስታነቡ በጣም በዳሰሳ ወይም በጥልቀት የታለፉ ክፍሎችን በግብረ መልስ መልክ ታሳውቁኛላችሁ ብየ ተስፋ አደርጋለሁ ፤ ይህን ማድረግ የመፅሐፉ ዋጋ ነው ብለን እንቁጠር ።
\vskip 0.3in
\noindent
\huge ማስታወሻ ተርጓሚው
\normalsize
\vskip 0.3in
\noindent
ተክኖሎጂ ነክ የሆኑ ሃሳቦችን በአማርኛ መፃፍ እጅጉን የከበደ ነገር እንደሆነ ትረዳላችሁ ብየ ተስፋ አደርጋለሁ ። ምክንያቱ እንደምረዳው ፤ ቋንቋችን ከተክኖሎጂ ጋር አብሮ ስላላደገ ነው ። ቋንቋችንን እያሳደግን ተክኖሎጂን ሁሉም ሰው መረዳት እንዲችል የቋንቋ ግርግዳውን ማፍረስ አለብን ። ለዚህ ሃሳብ አንድ ጠብታ ለማበርከት ነው እንደምንም ተፍጨርጭረ ይህንን አጭር መፅሐፍ ለመተርጎም የተነሳሁት ። ተክኖሎጂን ወደ ቋንቋችን አምጥተን የራሳችን ካላደረግነው እና በፈጠራ ውስጥም እኩል ተሳታፊ ካልሆንን በስተቀር ከመሰረታዊ ተክኖሎጂ ተጠቃሚነት አልፈልን በራሳችን ተክኖሎጂ የራሳችንን ችግር እንፈታለን ብየ አላምንም ። 
\vskip 0.3in
\noindent
\huge መጀመሪያ
\normalsize
\vskip 0.3in
\noindent


\begin{lstlisting}
package main

import "fmt"

func main() {
    fmt.Println("Hello World!")
}
\end{lstlisting}
\end{document}