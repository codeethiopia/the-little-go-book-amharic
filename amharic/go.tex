\documentclass[12pt,a4paper]{book}
\usepackage[utf8]{inputenc}
\usepackage{listings}
\usepackage{listings-golang} % import this package after listings
\usepackage{color}
\lstset{ % add your own preferences
    frame=single,
    basicstyle=\footnotesize,
    keywordstyle=\color{red},
    numbers=left,
    numbersep=5pt,
    showstringspaces=false, 
    stringstyle=\color{blue},
    tabsize=4,
    language=Golang % this is it !
}
\usepackage{fontspec}
\begin{document}
\fontspec{AbyssinicaSIL}
\noindent
\huge 
ስለዚህ መጽሐፍ
\normalsize
\vskip 0.5in
\huge 
\noindent
ፍቃድ
\vskip 0.3in

\normalsize
\noindent
The Little Go Book  በ Attribution-NonCommercial-ShareAlike 4.0 አለም አቀፍ ፍቃድ የተዘጋጀ ነው። ለዚህ መጽሐፍ ምንም አይነት ክፍያ መፈጸም የለበትም ። ይሄንን መጽሐፍ በነጻ መባዛት ፣ ማሰራጨት ፣ ይዘቱን መቀየር እንዲሁም ማስነበብ ይቻላል ። ነገር ግን ለጸሃፊው ካርል ሰጉዊን እውቅና መስጠት ይኖርቦታል ፤ መጽሐፉን ለንግድ አላማ አይጠቀሙበት።
\vskip 0.3in
\noindent
The Little Go Book  በ Attribution-NonCommercial-ShareAlike 4.0 አለም አቀፋዊ ፍቃድ የተዘጋጀ ነው። ለዚህ መጽሐፍ ምንም አይነት ክፍያ መፈጸም የለበትም ። ይሄንን መጽሐፍ በነጻ መባዛት ፣ ማሰራጨት ፣ ይዘቱን መቀየር እንዲሁም ማስነበብ ይቻላል ። ነገር ግን ለጸሃፊው ካርል ሰጉዊን እውቅና መስጠት ይኖርቦታል ፤ መጽሐፉን ለንግድ አላማ አይጠቀሙበት።
የመጽሐፉን ፍቃድ ሙሉ ዝርዝር የሚከተለው ማስፈንጠሪያ ላይ ማየት ይቻላል፤
https://creativecommons.org/licenses/by-nc-sa/4.0/
\vskip 0.3in
\noindent
\huge የመጨረሻ ስሪት
\normalsize
\vskip 0.3in
\noindent
የመጽሐፉን የቅርብ ጊዜ ምንጭ ከዚህ ማግኘት ይቻላል ፤ 

https://github.com/karlseguin/the-little-go-book
\vskip 0.3in
\huge
\noindent
መግቢያ
\normalsize
\vskip 0.3in
\noindent
ሁሌም አዲስ የፕሮግራሚንግ ቋንቋ ለመማር ስነሳ ድብልቅልቅ ስሜት ነው የሚሰማኝ ። በአንድ በኩል ፤ ቋንቋ (ፕሮግራሚንግ) ለምንሰራው ነገር በጣም መሰረታዊ ነው ፤በጣም አነስተኛዋ ለውጥ በጣም ከፍተኛ የሚባል ተጽእኖ ልትፈጥር ትችላለች። ፕሮግራም ስንጽፍ የሚገጥሙን አስተማሪ አጋጣሚዎች ወደፊት ለሚኖረን የፕሮግራሚንግ ስልት ዘላቂ የሆነ ተጽእኖ ሊፈጥሩብን እንዲሁም ከሌሎች ቋንቋዎችም የምንጠብቃቸውን ነግሮች እንድናገናዝብ ያደርጉናል። በሌላ በኩል ደግሞ የቋንቋዎች ዲዛይን በአብዛኛው ጭማሪ ጽንሰ-ሃሳቦችን ይዘው ነው የሚመጡት ። ስለዚህ አዲስ ቋንቋ ለመማር ፤ አዳዲስ ቁልፍ ቃላትን፣ የአይነት ስርዓት ፣ የኮድ ስልት ፣ ላይብረሪዎች ፣ ኮሚኒቲዎች እንዲሁም አዲስ አስተሳሰብ መለማመድ ከባድ ስራ ሊመስል ይችላል። ከሌሎች ትምህርቶች አንጻርም አዲስ ቋንቋ መማር ላይ የምናጠፋው ጊዘ የባከነ መስሎ ሊሰማን ይችላል።
\vskip 0.3in
\noindent
ይህን ካልኩ በኋላ ወደፊት ስንቀጥል ፤ አዲስ ቋንቋ በመማር ደረጃ በደረጃ ጭማሪ ለውጦችን ለመውሰድ ፍቃደኛ መሆን አለብን ፤ ምክንያቱም የፕሮግራሚንም ቋንቋዎች ለምንሰራቸው ነገሮች መሰረት ናቸው ። ለውጦቹ ቅጥልጣይ ጭማሪዎች ቢሆኑም ፤ ሰፊ አድማስ አላቸው ። በዚህም ምክንያት ምርታማነት (productivity) ፣ ተነባቢነት (readability) ፣ ውጠታማነት (performance) ፣ ተሞካሪነት (testability) ፣ የኮድ ጥገኝነት (dependency management) ፣ ጥርሰት አያያዝ (error handling) ፣ አሰናነድ (documentation) ፣ ትንታነ (profiling) ፣ ማህበረሰብ (communities) ፣ መነሻ ላይብረሪ (standard libraries) እና የመሳሰሉት ላይ ተጽእኖ ያሳርፋሉ። ምናልባትም እንደዚህ አይነት ለውጦችን እንደመራራ መድሐኒት ልንወስድ እንችላለን?
\vskip 0.3in
\noindent
ይህ ወደዋናው ጥያቀ ያመራናል ፤ ጎ ለምን አስፈለገ ? በእነ እይታ ሁለት አንገብጋቢ ምክንያቶች አሉ ። የመጀመሪያው ምክንያት ፤ በአንፃሩ ቀላል ቋንቋ ስለሆነና መነሻ ላይብረሪውም ቀላል ስለሆነ ነው ። በብዙ መልኩ ሲታይ ፤ ጎ ላይ ያሉት ለውጦች ባለፉት ሁለት አስርት አመታት በፕሮግራሚንግ ቋንቋዎች ላይ እየጨመረ የነበረውን ውስብስብነት የሚያቃልሉ ናቸው ። ለላው ምክንያት ደግሞ ለብዙ ገንቢዎች (developers) ጎ ተጨማሪ አቅም ስለሚሰጥ ነው ።
\vskip 0.3in
\noindent
ጎ የተገነባው እንደ ሲስተም ቋንቋ (ምሳለ ፦ ኦፐረቲንግ ሲስተም ፣ ዲቫይስ ድራይቨር) የ C እና የ C++ ገንቢዎችን ታሳቢ አድርጎ ነው ። ነገር ግን ጎ-ቲም (Go team) ባለው መረጃ መሰረት የ ጎ ዋና ተጠቃሚዎች የሆኑት የሲስተም ገንቢዎች ሳይሆኑ የ ትግበራ(application) ገንቢዎች ናቸው ። ለምን ? የሲስተም ገንቢዎችን ወክየ መናገር አልችልም ፤ ነገር ግን እንደነ ድረ-ገጽ ለሚገነቡ ፣ ለግልጋሎት (services) ፣ ለደስክቶፕ ትግበራዎች እና ለመሳሰሉት እየተፈጠረ ያለውን በከፍተኛ-እርከንና (higher-level) በዝቅተኛ እርከን (low-level) የሲስተም ትግበራዎች መካከል ያሉ ሲስተሞች ተፈላጊነት እያሟላ ስለሆነ ይመስለኛል ። 
\vskip 0.3in
\noindent
ምናልባትም ለመላላኪያ (messaging) ፣ ማቆያ (caching) ፣ ከፍተኛ ስለት ለሚያስፈልገው ውህብ ትንታነ ፣ ለትእዛዝ መስመር መግቢያ (command line interface) ፣ መመዝገቢያ (logging) ወይም መቆጣጠሪያ (monitoring) የተመቸ ቋንቋ ነው ልንል እንችላለን ። 




\begin{lstlisting}
package main

import "fmt"

func main() {
    fmt.Println("Hello World!")
}
\end{lstlisting}
\end{document}